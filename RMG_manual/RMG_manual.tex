\documentclass[12pt,letterpaper]{article}

\usepackage{pslatex}
\usepackage{amsmath}
\usepackage{amsfonts}
\usepackage{amssymb}
\usepackage{multirow}
\usepackage{url}
\usepackage{hyperref}
\begin{document}

\title{RMG Instruction Manual}
\author{Green Group \\ Massachusetts Institute of Technology}
\date{\today}
\maketitle
\pagenumbering{roman}

\newpage

\tableofcontents
\setcounter{tocdepth}{4}

\newpage
\setcounter{page}{0}
\pagenumbering{arabic}

\section{Introduction}
RMG is an automatic reaction mechanism generator. It is an object-oriented
program written in Java, which provides a stable, robust programming architecture
that is easily extended, modified, and improved. The core of RMG
is two data structures: graphs and trees. The graphs represent the chemical
structures, and the trees represent the databases of thermodynamic and
kinetic data.\\

RMG was developed by Dr. Jing Song under the guidance of Prof.
William Green in the Department of Chemical Engineering at the Massachusetts
Institute of Technology.
%%%%%%%%%%%%%%%%%%%%%%%%%%%%%%%%%%%%%%%%%%%%%%%%%%%%%%%%%%%%%%%%%%%%%%%%%%%%%%%%
\section{Installing RMG}

\subsection{Installation and License Requirements} \label{install requirements}
RMG is intended to be an open source program, available to the general
public free of charge. Currently, that is not the case. RMG does require
that the user obtain a license for some of RMG's dependencies, and \textbf{it is the
user's responsibility to ensure these licenses have been obtained}.

To install RMG, you will need:

\begin{enumerate}
\item The JAVA 2 Platform, Standard Edition. \\ \url{http://java.sun.com/j2se/index.jsp}
\item The DASPK3.1 solver package, which is available for research purposes but \underline{is} subject to copywrite restrictions. \\ \url{http://www.engineering.ucsb.edu/~cse/software.html}
\end{enumerate}
\subsection{Linux Installation}\label{linux installation}
To install RMG on a linux computer:  {\bf SANDEEP:  Are you going to create a makefile?  How do we deal with compiling DASPK once it has been untarred?}

\begin{enumerate}
\item Create the directory in which you want to install RMG.\\
\texttt{\$ mkdir <dir>} \\
For the purposes of this document, we will set the directory to:\\  \texttt{\$ mkdir /usr/share/RMG}

\item Set the RMG environment variable as the directory created in the first step:\\
\texttt{\$ RMG="<dir>"} \\
In our example, you would code:  \texttt{\$ RMG="/usr/share/RMG"}

\item Download the \texttt{RMG.tar}, untar it, and move it to the \$RMG directory:\\
\texttt{\$ tar -xvf RMG.tar} \\
\texttt{\$ mv RMG/* \$RMG}

\item  Before running RMG, a shared library file has to be generated which RMG will use to perform numerical integration.  Currently RMG uses DASPK3.0 (see Section \ref{install requirements}).  Copy the file \texttt{daskpk31.tgz} to the directory \texttt{"\$RMG/lib"} and untar it.\\
\texttt{\$ tar -zxvf daspk31.tgz}
\\ NOW WHAT?!

\item If you want to rum RMG from the bash shell, use your preferred editor and add the following lines to your \texttt{.bashrc} file: \\
\texttt{RMG="<dir>"} (e.g. \texttt{RMG="/usr/share/RMG"} )\\
\texttt{LD\_LIBRARY\_PATH=\$LD\_LIBRARY\_PATH:\$RMG/lib}

\end{enumerate}

\subsection{Windows Installation}\label{windows installation}
To install RMG on a Windows computer, unzip the \texttt{RMG.zip} file to a directory of your choice (e.g. \texttt{C://Programs/RMG}).\\
Once RMG is unzipped, you will need to create the \texttt{daspk.dll} library.  The file \texttt{daspk.dll} can be created using either Compaq Visual Fortran or Microsoft Visual C++.
\\ {\bf is the following valid only for MS Visual C++?  Should we drop the CV Fortran?}
\begin{enumerate}
\item Create a new empty Win32 Dynamic-Link Library project called daspk.
\item Go to project $\rightarrow$ add to project $\rightarrow$ files.
\item Add the following DASPK files (located in the DASPK3.0 directory):\\
\begin{tabular}{llll}
\texttt{adf\_dummy.f} & \texttt{daux.f}  & \texttt{dbanpre.f}  & \texttt{ddaspk.f} \\
\texttt{dilupre.f} & \texttt{dlinpk.f}  & \texttt{drbdpre.f}  & \texttt{drbgpre.f} \\
\texttt{dsensd.f} & \texttt{dsparsk.f}  & \texttt{mpi\_dummy.f}  &
\end{tabular}

\item Add the following C++ files (located in the /RMG/software/CPlusWrapper directory):\\
\begin{tabular}{llll}
\texttt{cdaspk.h} & \texttt{cdaspk.cpp}  & \texttt{JDASPK.cpp}  & \texttt{jing\_rxnSys\_JDASPK.h} \\
\texttt{parameter.h} & \texttt{parameter.cpp}  & \texttt{reaction.h}  & \texttt{reaction.cpp}
\end{tabular}

\item Go to Build $\rightarrow$ Build \texttt{daspk.dll}

\item Move \texttt{daspk.dll} to the \texttt{RMG/software/rmg} directory.

\end{enumerate}
\subsection{ChemGreen Installation}

TO BE ADDED

%%%%%%%%%%%%%%%%%%%%%%%%%%%%%%%%%%%%%%%%%%%%%%%%%%%%%%%%%%%%%%%%%%%%%%%%%%%%%%%%
\section{Creating an Initialization File}
To run RMG, you must specify the initial conditions and the model generation
parameters, all of which are contained in a single file, \texttt{condition.txt}. If you
are running RMG on a linux computer, the file \texttt{condition.txt} can be located
in any directory, although it is advised that you create a special directory
for RMG simulations, since each \texttt{condition.txt} creates its own subdirectories.
If you are using a Windows machine, the \texttt{condition.txt} must remain in
\texttt{RMG/condition} (please consult Section \ref{running in windows}).\\
The \texttt{condition.txt} file should specify the following conditions (in order):

\begin{enumerate}
\item Whether to restart a prior run
\item The main database
\item The primary thermodynamic library
\item The initial temperature and pressure of the system
\item The name, concentration, and chemical structure of the reactants
\item The specification and concentration of the inert gas(es)
\item The reaction families selected for modeling the system
\item The model enlarger
\item The finish controller
\item The dynamic simulator
\item Sensitivity analysis
\item The primary reaction library
\end{enumerate}

For Linux operation, the name \texttt{condition.txt} is variable. As you'll see in
Section \ref{running in linux}, you can change the name of the initialization file, so long as it
remains a \texttt{.txt} file. For Windows operation, the name of the file must remain
\texttt{condition.txt}.

\subsection{Restart}\label{restart}
The first item in the file \texttt{condition.txt} is the restart command. Whenever
RMG runs the \texttt{condition.txt} file, it creates a new directory (in the same directory
as the initialization file): \texttt{./restart/}. There are five text files within
the new restart directory: \texttt{allSpecies.txt}, \texttt{coreSpecies.txt}, \texttt{edgeSpecies.txt}, \texttt{coreReactions.txt}, and \texttt{edgeReactions.txt}. See Section \ref{RMG output} for a description of
these files.\\

There are two options for the restart command: \texttt{true} and \texttt{false}.\\

\texttt{Restart: true} \\

When the restart is set to true, RMG will read the five files in the \texttt{./restart/}
directory and resume the simulation. The restart command can be very
useful. If RMG ever crashes, it is possible to rerun the simulation. Similarly,
with the restart command you can run a simulation under a given set of
conditions and then rerun the same simulation with different conditions (e.g.
a new temperature, reactant concentration, etc.). Alternatively, you can use
the restart to change which species are examined in the sensitivity analysis.
If the simulation is new, then the restart command must be set to:\\

\texttt{Restart: false}

\subsection{Database}

In the main RMG directory is a directory of databases: \texttt{\$RMG/databases/}.
By default, there is one directory within the directory of databases: \texttt{\$RMG/databases/database}.
It is possible, however, to create multiple databases. For example, you
can copy the entire contents of the default database into a new database,
\texttt{\$RMG/databases/databaseII}, and change some of the values.
Unless you have created your own databases, the Database setting should
be left at the default value: \\

\texttt{Database: database}

\subsection{Primary Thermo Library}\label{primary thermo library}

By default, RMG will calculate the thermodynamic properties of the species
from Benson additivity formulas. In general, the group-additivity results are
suitably accurate. However, if you would like to override the default settings,
you may specify the thermodynamic properties of species in the primaryThermoLibrary.
When a species is specified in the primaryThermoLibrary, RMG
will automatically use those thermodynamic properties instead of generating
them from Benson's formulas. Multiple libraries may be created, if so desired.
Please see Section \ref{editing thermo database} for details on editing the primary thermo library.
In general, it is best to leave the PrimaryThermoLibrary set to its default
value:\\

\texttt{PrimaryThermoLibrary: primaryThermoLibary}

\subsection{Initial Conditions}

The fourth item in the initialization file is the initial temperature and pressure.
Currently, RMG can only model constant temperature and volume
systems. Future versions will allow for variable temperature and pressure.
Please note that the temperature and pressure must be accompanied by a
unit (in parentheses). Suitable temperature units are: K, F, and C. Suitable
pressure units are: atm, torr, pa, and bar. The following example assumes
that the system begins at 600 Kelvin and 200 atmospheres. \\

\begin{tabular}{l}
\texttt{TemperatureModel: Constant (K) 600} \\
\texttt{PressureModel: Constant (atm) 200}
\end{tabular}

\subsection{Reactants}

The name, concentration, and structure of each reactant must be specified.
For each reactant, the first line should include the reactant number (e.g.
(1)), its name (e.g. C2O2), and it's concentration with units (e.g. 4.09e-3
(mol/cm3)). The next line defines the reactant structure, described by an
adjacency list. ChemGreen is a useful tool for generating adjacency lists.
Please note that you may choose the simplified adjacency list in which the
hydrogen atoms are omitted (shown below). \\


\begin{tabular}{lllll}
\multicolumn{5}{l}{\texttt{InitialStatus:}} \\
\multicolumn{5}{l}{\texttt{(1) C2O2. 4.09e-3 (mol/cm3)}} \\
\texttt{1}   & \texttt{C} & \texttt{0} & \texttt{\{2,D\}} & \texttt{\{3,D\} }\\
\texttt{2}   & \texttt{C} & \texttt{0} & \texttt{\{1,D\}} & \texttt{\{4,D\} } \\
\texttt{3}   & \texttt{O} & \texttt{0} & \texttt{\{1,D\}} &\\
\texttt{4}   & \texttt{O} & \texttt{0} & \texttt{\{2,D\}} & \\
& & & & \\
\multicolumn{5}{l}{\texttt{(2) O2 1.1E-7 (mol/cm3) }}\\
\texttt{1} & \texttt{O} & \texttt{0} & \texttt{\{2,D\}} & \\
\texttt{2} & \texttt{O} & \texttt{0} & \texttt{\{1,D\}} & \\
\texttt{END} \\
\end{tabular} \\


Please note that the keyword ``\texttt{END}" must be placed at the end of the
InitialStatus section.\\

If one of the reactants is a resonance isomer, you only need to define one
of the resonance structures, and RMG will automatically detect the others.

\subsection{Inert Gases}

Following InitialStatus, the initialization file specifies which inert gases, if any,
are used. Currently RMG can handle three inert gases: N$_2$, Ne, and Ar. If
one of the gases is not used, set the concentration to 0.0 (mol/cm3). If no
bath gas is used, set all concentrations to zero. In the example below, there
is no nitrogen, the neon is omitted, and the argon concentration is 2.21E-6
(mol/cm3).\\

\begin{tabular}{l}
\texttt{InertGas:} \\
\texttt{N2 0.0 (mol/cm3)}\\
\texttt{Ar 2.21E-6 (mol/cm3)} \\
\texttt{END}
\end{tabular}\\

Please note that the keyword ``\texttt{END}" must be placed at the end of the
InertGas section.
\subsection{Reaction Model Enlarger}

The Reaction Model Enlarger specifies the rules for reaction mechanism
growth. Currently there is only one setting for mechanism growth, so please
leave this entry as it is: \\

\texttt{ReactionModelEnlarger: RateBasedModelEnlarger} \\

As RMG expands, other model enlargers will become available (e.g. for
pressure-dependent reaction networks).

\subsection{Finish Controller}

The Finish Controller defines the termination rules for model growth. Additionally,
it includes the precision parameters such as reaction time, conversion,
and error tolerance. Currently, there is only one finish controller. Future
versions of RMG will have additional finish controllers, such as pressure dependence.\\

Below the Finish Controller line, you must specify the goal. The goal tells
the program when to terminate the reaction model expansion. There are two
goal options: \texttt{ReactionTime} and \texttt{Conversion}. \texttt{ReactionTime} must be followed
by a number and a time unit. \texttt{Conversion} must be followed by a reactant
species name and a number between 0 and 1. \\

Beneath the goal, you must specify the error tolerance. Typical values
for the error tolerance are between 0.0001 and 0.5. A smaller error tolerance
corresponds to a larger model; consequently, it is advised to use a large
error tolerance when generating a new model. Once the initial simulation
has finished, you can tighten the tolerance and apply the restart function to
generate an expanded model.
In the first example, the reaction will run for 0.001 seconds. In the second
example, the reaction will run until 90\% of the C$_2$O$_2$ is consumed and has a
much tighter error tolerance. \\

\begin{tabular}{l}
\texttt{FinishController: RateBasedFinishController} \\
\texttt{(1) Goal ReactionTime: 0.001 (sec)} \\
\texttt{(2) Error Tolerance: 0.1} \\
\end{tabular}\\

or\\

\begin{tabular}{l}
\texttt{FinishController: RateBasedFinishController} \\
\texttt{(1) Goal Conversion: C2O2 0.90} \\
\texttt{(2) Error Tolerance: 0.0001} \\
\end{tabular}\\

For the initial model run, it is often preferable to use \texttt{Goal Conversion}
instead of \texttt{Goal Reaction Time}, since it is difficult to judge \textit{a priori} what the
reaction time should be.\\

For a more detailed description on rate-based model enlargement, please
consult the paper ``Rate-Based Construction of Kinetic Models for Complex
Systems" by Roberta G. Susnow, Anthony M. Dean, William H. Green,
Pawel K. Peczak, and Linda J. Broadbelt, in the \textit{Journal of Physical Chemistry
A}, Vol. 101, pages 3731-40, 1997.

\subsection{Dynamic Simulator}

RMG uses the DASPK dynamic solver. Future versions of RMG might include
the CHEMKIN dynamic solver. \\

The first line beneath the solver is \texttt{TimeStep}, which is followed by a number
and a unit of time. The \texttt{TimeStep} indicates how frequently RMG should
check to see if the model needs to be enlarged. In the current example, RMG
will check every second of simulation time to see whether the error tolerance
is satisfied. The next two lines specify the absolute and relative tolerance,
respectively. Common values for the absolute tolerance are 1e-15 to 1e-25.
Relative tolerance is usually 1e-3 to 1e-5. \\

\begin{tabular}{l}
\texttt{DynamicSimulator: DASPK} \\
\texttt{TimeStep: 1 (sec)} \\
\texttt{Atol: 1e-20} \\
\texttt{Rtol: 1e-4} \\
\end{tabular}\\


\subsection{Sensitivity Analysis}

As long as DASPK is chosen for the dynamic solver, RMG can perform
sensitivity analysis and generate error bars on the concentration profiles.
The sensitivity analysis is based upon first-order sensitivity coefficients and
uncertainties in the rate constants. To utilize the sensitivity analysis, either
the first or second line must be set to \texttt{on}. If both \texttt{Error bars} and \texttt{Display sensitivity coefficients} are set to off, then RMG will not perform a sensitivity
analysis.\\

When the option to generate error bars is turned on, RMG calculates
the upper and lower bounds on the concentration profiles for all core species.
These upper and lower bounds are generated by the first-order sensitivity
equation:\\

\begin{equation}
\ln C_i \left( t,\overrightarrow{k} + \Delta \overrightarrow{k} \right) \approx \ln C_i \left( t,\overrightarrow{k} \right) + \sum_j \frac{\partial \ln C_i}{\partial \ln k_j} \Delta \ln k_j \nonumber
\end{equation} 

where $C_i$ is the concentration of the $i^{\underline{th}}$ species, and $\overrightarrow{k}$ is the vector of rate
constants.\\

If the option to display sensitivity coefficients is turned on, you must
specify a list of species for which the sensitivities should be displayed. The
output file will then include the sensitivities of those species with respect to
all the rate constants, all the heats of formations, and the initial concentrations
of all the reactants. Additionally, if you choose to display the sensitivity
coefficients, RMG will print a list of the five reactions that contribute the
most to the uncertainty in the concentration (for each species selected by the
user).\\

Sensitivities to rate constants are normalized sensitivities ($\partial \ln C_i/ \partial \ln k_j$),
and sensitivities to heats of formation are semi-normalized ($ \partial \ln C_i/ \partial \Delta H_f$ ).
The contribution of a reaction to the uncertainty is estimated as the product
of the normalized sensitivity and the uncertainty in the rate constant ($\Delta \ln k$).
In the following example, RMG will perform a sensitivity analysis. Error
bars will be generated for all the species, and the sensitivity coefficients will
be generated for carbon monoxide, carbon dioxide, methane, and water.\\

\begin{tabular}{l}
\texttt{Error bars: On} \\
\texttt{Display sensitivity coefficients: Off} \\
\texttt{Display sensitivity information for:} \\
\texttt{CO} \\
\texttt{CO2} \\
\texttt{CH4} \\
\texttt{H2O} \\
\texttt{END} \\
\end{tabular}\\

Please note that the keyword ``\texttt{END}" must be placed at the end of the
Error bars section.

\subsection{Primary Reaction Library}

The final section of the file \texttt{condition.txt} specifies which, if any, primary reaction
libraries are used. Currently, the only available reaction library included
with RMG is the Leeds' oxidation library. You can specify your own primary
reaction library in the location section. Please note that the oxidation library
should not be used for pyrolysis models. In the following example, the user
has created her own reaction library with a few additional reactions specific
to n-butane, and these reactions are to be used in addition to the Leeds'
oxidation library.\\

\begin{tabular}{l}
\texttt{PrimaryReactionLibrary: on}\\
\texttt{Name: nbutane}\\
\texttt{Location: database/primaryReactionLibrary/nbutane}\\
\texttt{Name: Leeds}\\
\texttt{Location: database/primaryReactionLibrary/combustion\_core/version5}\\
\texttt{END}
\end{tabular}\\


Please note that the keyword ``\texttt{END}" must be placed at the end of the
Primary Reaction Library section. Because the units for the Arrhenius parameters
are given in each library, the different libraries can have different units.
Additionally, if the same reaction occurs more than once in the combined
library, the first instance of it from the first library in which it appears is the
one that gets used.

%%%%%%%%%%%%%%%%%%%%%%%%%%%%%%%%%%%%%%%%%%%%%%%%%%%%%%%%%%%%%%%%%%%%%%%%%%%%%%%%
\section{Modifying RMG}
\subsection{Editing the Thermodynamic Database}\label{editing thermo database}
As mentioned in Section \ref{primary thermo library}, it is possible to override the default thermodynamic
values and substitute your own thermodynamic data.

\subsubsection{Basic Structure of the Thermo Database}

The thermodynamics database consists of three sections, each of which is an ASCII file that can be edited to alter the information. This description
applies to non-radical groups. There are other tree/library/dictionary files
for radical groups, ring corrections, and other corrections. The nomenclature
is different, but they would be edited in a similar way. The files of interest
are located in the directory: \texttt{\$RMG/databases/database/thermo/}.
The three files are described below. \\

\textbf{Dictionary File} (\texttt{Group\_Dictionary.txt}): This file contains the name and
adjacency list (structure) for all of the nodes contained within the thermo
tree. The nomenclature within all three files must be identical. The asterisk
``*" denotes the central atom in the group for which the group value is defined.
The format for each line in the adjacency list is: atom \#, * if central atom,
element, \# of radicals (0, 1, 2) on element, bonding.\\

\begin{tabular}{llllll}
\multicolumn{6}{l}{\texttt{Cb-(Cds-Cdd-Cd)}} \\
\texttt{1} & \texttt{*} & \texttt{Cb}  & \texttt{0} & \texttt{\{2,S\}} &  \\
\texttt{2} &            & \texttt{Cd}  & \texttt{0} & \texttt{\{1,S\}} & \texttt{\{3,D\}} \\
\texttt{3} &            & \texttt{Cdd} & \texttt{0} & \texttt{\{2,D\}} & \texttt{\{4,D\}} \\
\texttt{4} &            & \texttt{C}   & \texttt{0} & \texttt{\{3,D\}} & \\
\end{tabular}\\

\textbf{Tree File} (\texttt{Group\_Tree.txt}): This file defines the tree structure of the
database. The nodes at any particular level are defined by including the text
``\texttt{Lx:}" prior to the name of the group at that node, where ``\texttt{x}" is a number
corresponding to the level in the tree. A sample is given below:\\

\begin{tabular}{llllllll}
\texttt{L0:} & \texttt{R} & & & & & & \\
             & \texttt{L1:} & \texttt{C} & & & & & \\
             &              & \texttt{L2:} & \texttt{Cbf} & & & & \\
             &              &              & \texttt{L3:}  & \multicolumn{3}{l}{\texttt{Cbf-CbCbCbf}} & \\
             &              & \texttt{L2:} & \texttt{Cb}  & & & & \\
             &              &              & \texttt{L3:}  & \multicolumn{2}{l}{\texttt{Cb-H}} & & \\
             &              &              & \texttt{L3:}  & \multicolumn{2}{l}{\texttt{Cb-Os}} & & \\
             &              &              & \texttt{L3:}  & \multicolumn{2}{l}{\texttt{Cb-C}} & & \\
             &              &              & & \texttt{L4:}  & \multicolumn{2}{l}{\texttt{Cb-Cs}} & \\
	     &              &              & & \texttt{L4:}  & \multicolumn{2}{l}{\texttt{Cb-Cds}} & \\
	     &              &              & & & \texttt{L5:}  & \multicolumn{2}{l}{\texttt{Cb-(Cds-Od) // Cb-CO}} \\
	     &              &              & & & \texttt{L5:}  & \multicolumn{2}{l}{\texttt{Cb-(Cds-Cd)}} \\
	     &              &              & & & & \texttt{L6:}  & \texttt{Cb-(Cds-Cds) // Cb-Cd} \\
             &              &              & & \texttt{L4:}  & \multicolumn{2}{l}{\texttt{Cb-Cb}} & \\
             &              & \texttt{L2:} & \texttt{Ct} & & & & \\
\end{tabular} \\

Note that the indentation is not necessary because it is the ``\texttt{Lx:}" that
the computer reads, but is very helpful in making these files human readable.\\

\textbf{Library File} (\texttt{Group\_Library.txt}): This is the archive of the actual data
associated with a given group in the tree and dictionary. There are 15 (space
or tab separated) fields in the thermo library to describe the group. The
units for enthalpy are kcal/mole, and the units for entropy and heat capacity
(Cp) are cal/mole-K. The columns are described in Table \ref{library table}.\\

\begin{table}
\begin{tabular}{|l|p{11.1cm}|} \hline
Column \# & What it contains \\ \hline
1     & A unique number; this does not correspond to any other part of the thermo database, but numbering sequentially is most logical \\ \hline
2     & Group name; same as in tree and dictionary \\ \hline
3-4   & Enthalpy and Entropy at 298K \\ \hline
5-11  & Cp at T = 300, 400, 500, 600, 800, 1000, and 1500K \\ \hline
12-13 & dH and dS: absolute uncertainties in
the enthalpy/entropy estimates \\ \hline
14    & dCp: absolute uncertainty in the Cp estimate
(no temperature consideration) \\ \hline
15    & Comments Section: usually citing the source of
the data or comments on the reliability \\ \hline
\end{tabular} \\
\caption{Description of Library File Structure}
\label{library table}
\end{table}
A sample entry can be found in Section \ref{changing thermo data}.

\subsubsection{Thermo Database and Adjacency List Notation}

In general, the thermo database uses what are known as function group
elements. Function group elements serve to define the atom and its bonding
environment. These definitions serve to simplify the adjacency lists of groups
and allow for more general descriptions of groups. The notation used in the
database is shown below. New functional group elements cannot be added to
RMG in a simple way, as they must be hard-coded into the RMG software
with the appropriate properties.\\

If you examine the file \texttt{Group\_Dictionary.txt}, you will see that these groups
are used extensively, even more so that the actual atoms C, H, or O. You
will also see that groups can be defined using a bracketed notation, which
simply means that either atom/functional group element will generate this
node. For example:\\

\begin{tabular}{lllll}
\multicolumn{5}{l}{\texttt{Cb-Cds}} \\
\texttt{1} &  \texttt{*} &  \texttt{Cb} & \texttt{0} & \texttt{\{2,S\}} \\
\texttt{2} &             &  \texttt{\{Cd,CO\}} &  \texttt{0} & \texttt{\{1,S\}}
\end{tabular}

In this example, the second ``element" is actually either Cd or CO. The
reason that either Cd or CO will generate the same node is because both
fall under the more general Cds definition, which is a carbon atom with one
double and two single bonds. This can also be useful if there is a secondary
effect and all that matters is that there is a $\pi$-bond present, but the fact that
it is C=O or C=C does not matter. This is not used very much in
the thermo database, but occurs much more in the kinetics databases where
radical delocalization can play a major role in determining the rate constant
of a reaction. A table of the possible functional groups can be seen in Table \ref{functional group table}.

\begin{table}
\begin{tabular}{|l|c|p{9cm}|} \hline
Cs  & The graphics & Carbon bonded to four single bonds \\ \hline
Cd  & will be added & Carbon bonded to a double bond and two single bonds.  (The other end of the double bond is carbon) \\ \hline
Cdd & later & Carbon bonded to two double bonds \\ \hline
Ct  & & Carbon bonded to a triple bond and single bond \\ \hline
Cb  & & Carbon bonded to two benzene bonds and a single bond.  (The carbon belongs to only one benzene ring) \\ \hline
Cbf & & Carbon bonded to three benzene bonds (the carbon belongs to two or three benzene rings) \\ \hline
CO  & & Carbon bonded to a double bond and two single bonds.  (The other end of the double bond is oxygen) \\ \hline
Os  & & Oxygen bonded to two single bonds \\ \hline
Od  & & Oxygen bonded to a double bond \\ \hline
Oa  & & Oxygen triplet \\ \hline
R   & & Any atom \\ \hline
R!H & & Any non-hydrogen atom \\ \hline
\end{tabular}
\caption{Functional Group Elements}
\label{functional group table}
\end{table}

\subsubsection{Changing the Data for an Existing Group}\label{changing thermo data}

This task is relatively simple, as it only involves commenting out the line you
want to replace and putting new values in its place. Only the file \texttt{Group\_Library.txt}
will need to be altered in this case. Perhaps the simplest way to show this
is through an example. Commenting out a line can be accomplished by
introducing a ``\texttt{//}" at the start of the line.\\

Let's assume that you have the following group values, which are for a
carbon in a benzene ring attached to an H and to a single-bonded O:\\

{\tiny{
\begin{tabular}{lllllllllllllll}
\texttt{\#} & \texttt{Group} & \texttt{H} & \texttt{S} & \texttt{Cp300} & \texttt{Cp400} & \texttt{Cp500} & \texttt{Cp600} & \texttt{Cp800} & \texttt{Cp1000} & \texttt{Cp1500} & \texttt{dH} & \texttt{dS} & \texttt{dCp} & \texttt{Note} \\ \hline
\texttt{7} & \texttt{Cb-H} & \texttt{3.3} & \texttt{11.53} & \texttt{3.24} & \texttt{4.44} & \texttt{5.46} & \texttt{6.3} & \texttt{7.54} & \texttt{8.41} & \texttt{9.73} & \texttt{0.11} & \texttt{0.12} & \texttt{0.1} & \texttt{Cb-H Benson} \\
\texttt{8} & \texttt{Cb-Os} & \texttt{-0.9} & \texttt{-10.2} & \texttt{3.9} & \texttt{5.3} & \texttt{6.2} & \texttt{6.6} & \texttt{6.9} & \texttt{6.9} & \texttt{7.07} & \texttt{0.16} & \texttt{0.1} & \texttt{0.1} & \texttt{Cb-O Benson} \\
\end{tabular}\\
}}

Suppose you think that the group enthalpy for Cb-H is actually 3.85
kcal/mole instead of 3.3 kcal/mole, and you are quite sure, so the dH = 0.03 instead of 0.11. The suggested way
to change this would be to comment out (deletion is also an option, but
not advised) line 7 and add a similar line with the new values. It is also
suggested that you describe why you have alter the database in the "Note"
section, potentially providing a URL to a more detailed description. The
new scenario would look like this:\\

{\tiny{
\begin{tabular}{lllllllllllllll}
\texttt{\#} & \texttt{Group} & \texttt{H} & \texttt{S} & \texttt{Cp300} & \texttt{Cp400} & \texttt{Cp500} & \texttt{Cp600} & \texttt{Cp800} & \texttt{Cp1000} & \texttt{Cp1500} & \texttt{dH} & \texttt{dS} & \texttt{dCp} & \texttt{Note} \\ \hline
\texttt{//7} & \texttt{Cb-H} & \texttt{3.3} & \texttt{11.53} & \texttt{3.24} & \texttt{4.44} & \texttt{5.46} & \texttt{6.3} & \texttt{7.54} & \texttt{8.41} & \texttt{9.73} & \texttt{0.11} & \texttt{0.12} & \texttt{0.1} & \texttt{Cb-H Benson} \\
\texttt{7} & \texttt{Cb-H} & \texttt{3.85} & \texttt{11.53} & \texttt{3.24} & \texttt{4.44} & \texttt{5.46} & \texttt{6.3} & \texttt{7.54} & \texttt{8.41} & \texttt{9.73} & \texttt{0.03} & \texttt{0.12} & \texttt{0.1} & \texttt{My Value is so much better} \\
\texttt{8} & \texttt{Cb-Os} & \texttt{-0.9} & \texttt{-10.2} & \texttt{3.9} & \texttt{5.3} & \texttt{6.2} & \texttt{6.6} & \texttt{6.9} & \texttt{6.9} & \texttt{7.07} & \texttt{0.16} & \texttt{0.1} & \texttt{0.1} & \texttt{Cb-O Benson} \\
\end{tabular}\\
}}

You may also refer one group value to another if you believe that they
should be the same. This is done by putting the name of the group
that has the same thermo parameters in column 3 and leaving columns 4-
15 blank. Effectively, if RMG encounters the group and finds the name of
another group instead of numerical value, it will assign the current group the
same values that occur in the referred group.\\

\begin{tabular}{lllll}
\texttt{253} & \texttt{Cb-(Os-(Os-Cs))} & \texttt{-2.5} & \texttt{-8.5} & \texttt{...} \\
\texttt{254} & \texttt{Cb-(Os-(Os-H))} & \texttt{Cb-(Os-(Os-Cs))} &
\end{tabular} \\

The above example would assume that the group values for \#254 are
identical to those of \#253.\\

This referencing does not need to be done if the group to which you want
to refer lies directly above the current group in the tree, because if the tree
does not have a certain node defined it will look back up the tree to find the
nearest relative and use those values.\\


\begin{tabular}{lllll}
\texttt{253} & \texttt{Cb-(Os-(Os-Cs))} & \texttt{-2.5} & \texttt{-8.5} & \texttt{...} \\
\texttt{254} & \texttt{Cb-(Os-(Os-H))} & \texttt{Cb-(Os-(Os-Cs))} & \\
\texttt{255} & \texttt{Cb-(Os-(Os-(Cs-OsHH)))} & \texttt{Cb-(Os-(Os-Cs))} & \\
\end{tabular} \\


In this case, the referring of \#255 back to \#253 is unnecessary because
RMG would refer back to \#253 by default if it did not find any values for
\#255. This redundancy will not create any problems within RMG, but it is
simply unnecessary.\\

If you are adding a new group, you will need to follow the procedure
below for adding a group to the tree, dictionary, and library files as well.\\

Please be aware that version control may be a significant issue with the databases and should
be addressed early to ensure consistency within the group.

\subsubsection{Adding Additional Nodes to the Thermo Database}

This task is more complicated than the previous example because it involves altering all three database
files in a consistent manner. It is worth noting that the order in which
species are added to the dictionary and library does not matter; however,
the position where the new group is added to the tree is of the utmost
importance. It is useful to create the tree and dictionary with items in the
same order.  This ordering will facilitate cross checking and debugging if needed. Since all
files must use the same nomenclature, the ability to search may make consistent
ordering unnecessary. The procedure is described below.\\

\begin{enumerate}

\item Find the appropriate location to place the new group and ensure that
the nomenclature is unambiguous and unique. Placing the group in the
incorrect location could cause incorrect estimates to be made when the
tree is being searched. Using Chemgreen to navigate the tree structure
is a useful way to find the location for your new group.\\

\item Add the line of text to the \texttt{Group\_Tree.txt} file in the following form,
where ``\texttt{x}" is the appropriate level. The following example is for a C-O-O-H off of a
benzene ring.\\

\texttt{Lx: Cb-(Os-(Os-H))}\\

\item Using the same name as in the tree, append the file \texttt{Group\_Dictionary.txt} to define the structure of the group and its atom center (denoted by the *):

\begin{tabular}{llllll}
\multicolumn{6}{l}{\texttt{Cb-(Os-(Os-H))}} \\
\texttt{1} & \texttt{*} & \texttt{Cb}  & \texttt{0} & \texttt{\{2,S\}} &  \\
\texttt{2} &            & \texttt{O}  & \texttt{0} & \texttt{\{1,S\}} & \texttt{\{3,S\}} \\
\texttt{3} &            & \texttt{O} & \texttt{0} & \texttt{\{2,S\}} & \texttt{\{4,S\}} \\
\texttt{4} &            & \texttt{H}   & \texttt{0} & \texttt{\{3,S\}} & \\
\end{tabular}\\

\item Add the new group to the \texttt{Group\_Library.txt} file using the same nomenclature
and whatever thermo data you have for the group. The format was shown in the previous section.\\

\begin{tabular}{l}
{\small{\texttt{2374 Cb-(Os-(Os-H)) 3.5 10.0 ... "I added this b/c ..."} }}\\
\end{tabular}\\
\end{enumerate}

\subsection{Editing the Primary Reaction Library}

A primary reaction library (PRLib) consists of four files: \texttt{reactions.txt}, \texttt{species.txt},
\texttt{3rdBodyReactions.txt}, and \texttt{troeReactions.txt}. It is not necessary to define reactions
for each of the files if the user does not wish to, but all four files must
exist, and they must include the Unit declaration and Reaction headings.\\

When you use a PRLib, all species and reactions in the library are immediately
added to the mechanism, though the species come in with zero
initial concentration unless otherwise specified in the input condition file.
The user will see these reactions listed in the file \texttt{Final\_Model.txt}. PRLibs
are particularly useful for (i) small molecule reactions which tend to deviate
from the standard library values more significantly and (ii) to incorporate a
set of known experimental values for key reactions.\\

\begin{enumerate}

\item The first file, \texttt{reactions.txt}, defines the standard reactions. The file has
the structure shown in the following sample. Comments are denoted
with ``\texttt{//}" and are ignored by the RMG parser.\\

{\footnotesize{
\begin{tabular}{l}
\texttt{// Define the units} \\
\texttt{// Units allowed for A are: "mol/liter/s" or "mol/cm3/s"} \\
\texttt{// Units allowed for E are: "kcal/mol", "cal/mol", "kJ/mol", or "J/mol"} \\
\\
\texttt{Unit:} \\
\texttt{A: mol/cm3/s}\\
\texttt{E: kcal/mol}\\
\\
\texttt{Reactions:}\\
\texttt{// The format is:}\\
\texttt{// "R1 + R2 = P1 + P2 A n Ea dA dn dEa "}\\
\texttt{// where R1, R2, P1, P2 are species; A, n, and Ea are the Arrhenius }\\
\texttt{// parameters, and dA, dn, dEa are the errors in those parameters (normally} \\
\texttt{// additive, but can also be multiplicative if a * comes before the number) }\\
\texttt{O2 + CO = CO2 + O 1.26E13 0.00 196.90 *1.7 0 0 }\\
\end{tabular}\\
}}

\item The species list is just a series of molecular names and connectivity
lists, analogous to the formats used in the input file \texttt{condition.txt}. All
species present in \texttt{reactions.txt} must be given a structure in \texttt{species.txt}.
The structure should have the same format as the adjacency list shown
in Section 3.1.6. Please note that the names that are used in the
reaction library will be those used throughout the mechanism. In this
manner, the user can adopt a preferred nomenclature for individual
species.\\

\item The next file is \texttt{3rdBodyReaction.txt}, which lists reactions involving a
third body (bath gas). A sample file, with a 3rd-body reaction, is listed
below. The first line defines the reaction using the same format as in \texttt{reactions.
txt}. The next line lists collision efficiencies for various bath-gas
species that scale the concentrations of particular species when calculating
the total bath gas concentration. In the example below, CH$_4$ is
particularly effective as a 3rd body, and its effective concentration is
tripled.\\

{\footnotesize{
\begin{tabular}{l}
\texttt{// Define the units} \\
\texttt{// Units allowed for A are: "mol/liter/s" or "mol/cm3/s"} \\
\texttt{// Units allowed for E are: "kcal/mol", "cal/mol", "kJ/mol", or "J/mol"} \\
\\
\texttt{Unit:} \\
\texttt{A: mol/cm3/s} \\
\texttt{E: kJ/mol} \\
\\
\texttt{Reactions:} \\
\texttt{CO + O + M = CO2 + M 1.54E15 0.00 12.56 *1.2 0 0} \\
\texttt{N2/0.4/ O2/0.4/ CO/0.75/ CO2/1.5/ H2O/6.5/ CH4/3.0/ C2H6/3.0/ AR/0.35/} \\
\texttt{// the first line defines the reaction and Arrhenius parameters,} \\
\texttt{// while the second gives the scaling factors for different bath gas species} \\
\texttt{// which contribute to [M].} \\
\end{tabular}\\
}}

\item The final file, \texttt{troeReactions.txt}, is present for specifying pressure-dependent
reactions according to the Troe expression. The format of each reactions
is the same as is used by ChemKin. For example:

{\footnotesize{
\begin{tabular}{l}
\texttt{Unit:} \\
\texttt{A: mol/cm3/s} \\
\texttt{E: kJ/mol} \\
\\
\texttt{Reactions:} \\
\texttt{C2H2 + H (+M) = C2H3 (+M) 8.43E12 0.00 10.81 *1.2 0 0} \\
\texttt{N2/0.4/ O2/0.4/ CO/0.75/ CO2/1.5/ H2O/6.5/ CH4/3.0/ C2H6/3.0/ AR/0.35/} \\
\texttt{LOW / 3.43E18 0.0 6.15 /} \\
\texttt{TROE / 1 1 1 1231 /} \\
\\
\texttt{// the first two lines are very similar to the 3rdBodyReactions.txt format} \\
\texttt{// the next like specifies either the low pressure limit Arrhenius} \\
\texttt{// parameters, and the final line specifies the 3 or 4 Troe parameters} \\
\texttt{// in the order: a, T***, T*, T** (the last parameter is optional).} \\
\end{tabular}\\
}}









\end{enumerate}

%%%%%%%%%%%%%%%%%%%%%%%%%%%%%%%%%%%%%%%%%%%%%%%%%%%%%%%%%%%%%%%%%%%%%%%%%%%%%%%%
\section{Running RMG}
\subsection{Running RMG from the Command Line in Linux}\label{running in linux}

This section assumes that RMG is already installed according to the directions
in Section \ref{linux installation}. Additionally, it is assumed that you have created a
directory (either in your home directory or elsewhere) in which the initialization file \texttt{condition.
txt} has been placed. In the example, we assume that we have created a
directory \texttt{/hexane\_pyro}, which contains the condition file \texttt{hexane\_pyro.txt}.

\begin{enumerate}
\item Change the current working directory to the directory with the condition file:\\

\texttt{\$ cd /hexane\_pyro/}\\

\item Run the following Java command to launch the program (all one line):\\

\texttt{\$ java -Xmx500m -classpath \$RMG/software/RMG RMG hexane\_pyro.txt
> output \&} \\

When RMG is started, it creates six subdirectories: \texttt{chemkin}, \texttt{fit3p},
\texttt{GATPfit}, and \texttt{Restart}.\\

In this example, all of the output from RMG is piped to a file output. The
\texttt{> output} part of the command is optional but recommended. Please note
that the Java option \texttt{-Xmx500m} is hardware specific. In this example, we
assume that we can allot five hundred megabytes of memory for Java Virtual
Machine running RMG; please consult a guide to Java for more information.
To see the content of the output file as it is being generated, use the command:\\ \texttt{\$ tail -f output}.

\end{enumerate}

\subsection{Running RMG from the Command Line in Windows}\label{running in windows}

Running RMG in Windows is a little different than under linux. The most
important difference is the location and name of the condition file. In the
Windows version of RMG, there is a condition directory in the RMG directory:
\texttt{RMG/condition}. The \texttt{condition.txt} file must be placed inside this
directory, and it must have the title \texttt{condition.txt}.\\

To run RMG in Windows:

\begin{enumerate}
\item  Place the new \texttt{condition.txt} in \texttt{RMG/condition}\\

\item  Open a DOS-prompt\\

\item  Change directories to \texttt{RMG/software/rmg}\\

\item  type \texttt{run}
\end{enumerate}

%%%%%%%%%%%%%%%%%%%%%%%%%%%%%%%%%%%%%%%%%%%%%%%%%%%%%%%%%%%%%%%%%%%%%%%%%%%%%%%%
\section{Analyzing the Output}
\subsection{RMG Output Files}\label{RMG output}

RMG will generate several types of output files, each with a different purpose.
For linux users, these files will be located in the same directory which
contains the condition file; for Windows users, the output files are located
in the directory \texttt{RMG/software/RMG/}. Generally, there are five outputs from
RMG:

\begin{enumerate}
\item \texttt{output.txt}.  This is a large file that shows the many steps RMG went
through to build the model, including the addition of each new species,
concentrations and fluxes at all of the reported integration time steps, and generally the details as RMG builds the mechanism. It is useful
for debugging if a problem occurs, but other output files are better if
all you care about is the final model and its behavior. This file also
lists the input file at the beginning.\\

\item  \texttt{Final\_Model.txt}.  As its name would imply, this file contains the details of
the final model created by RMG. It gives the total number of reactions
and both the species in the model (core) and in the model's edge (those species
whose flux was not high enough to be included in the model). It also
gives all of the reactions included in the model, but does not list the
species directly. If you want the species list, see the Chemkin file or
the dictionary file (described below). After the list of reactions, the concentration
and mole fraction profiles of each species are listed, based on
the time step provided in the input file. This data can easily be pasted
into a spreadsheeting program for analysis and plotting. If sensitivity
information was requested, it would also be displayed in this file.\\

\item \texttt{chem.inp} and \texttt{therm.dat} (located in \texttt{./chemkin}).  These are the Chemkin model and thermodynamic data input files
of the model. The file \texttt{chem.inp} lists the elements, species, reactions,
and thermodynamic data (without N$_2$, Ar, or Ne) for the model. The
full list of thermodynamic data (with N$_2$, Ar, or Ne) can be found in
\texttt{therm.dat}.\\

\item \texttt{RMG\_Dictionary.txt}.  This file lists all the species by name
that are used in the model, as well as their adjacency list;  consequently, it will allow to user to see a graphical representation of all species in the model. Typically the user will copy this file to an appropriate directory so that it can be read by Chemgreen (See Section \ref{view chemgreen} for further details).\\

\item Restart files (located in \texttt{./Restart/}).  As mentioned in Section \ref{restart}, these five files are used if the job needs to be restarted.  The
user does not ever need to interact with these files directly, but they do
list more details than any of the other output files. The five files are:
\texttt{coreSpecies.txt}, \texttt{coreReactions.txt}, \texttt{edgeSpecies.txt}, \texttt{allSpecies.txt} and \texttt{all
Reactions.txt}. The file \texttt{coreSpecies.txt} lists the identification name/number
for all species in the core model, \texttt{edgeSpecies.txt} lists the edge species
identification name/number, and \texttt{allSpecies.txt} lists both the identification
name/number and the structural data for all species in both the core and edge. The reaction files are similar except they list reaction data.\\

\end{enumerate}

\subsection{Viewing the Species in ChemGreen}\label{view chemgreen}

When RMG has finished, it will create a file \texttt{RMG\_Dictionary.txt} in the same
directory as the \texttt{condition.txt} file. This file contains a list of all the core species
(usually the edge species are too numerous to be of use); the core species can
be viewed easily using ChemGreen. To view the core species, rename the first part of the file (this step isn't strictly necessary, but it is
convenient and a good practice). The file must still end in ``\texttt{\_Dictionary.txt}"
(e.g. \texttt{newname\_Dictionary.txt}) for ChemGreen to read it. Next, copy this file
to the directory \texttt{/\$ChemGreen/.../thermo}. Once the file is in the thermo directory,
double-click on the \texttt{chemgreen.bat} file to launch the program. Please
note that you must close and relaunch the program every time you add a
new file to the thermo directory. Once ChemGreen has launched, doubleclick
on the folder \texttt{Thermo}. Your new file should appear in this directory (e.g. newname without the ``\texttt{\_Dictionary.txt}").\\

A new window should appear with a list of names. Highlight any name
and click "view" to see its structure.
\end{document}
